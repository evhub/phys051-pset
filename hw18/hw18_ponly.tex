% HMC Math dept HW template example
% v0.04 by Eric J. Malm, 10 Mar 2005
\documentclass[12pt,letterpaper,boxed,cm]{hmcpset}

% set 1-inch margins in the document
\usepackage[margin=1in]{geometry}
\usepackage{mathtools}
\usepackage{mathrsfs}
% include this if you want to import graphics files with /includegraphics
\usepackage{graphicx}
\usepackage{cases}
\usepackage{hyperref}
\usepackage{siunitx}
\usepackage{tikz}
\usepackage{cases}
\usepackage{mathalfa}
\usetikzlibrary{arrows}

% info for header block in upper right hand corner
\name{Name: ~~~~~~~~~~~~~~~~~~~~~~~~~~~~~~~}
\class{Physics 51}
\assignment{Homework \#18}
\duedate{November 14, 2016}

\newcommand{\ev}[2]{\Big|_{#1}^{#2}}
\newcommand{\evv}[2]{\Big|_{#1}^{#2}}
\newcommand{\set}[1]{\left\{#1\right\}}
\newcommand{\s}[1]{\sqrt{#1}}
\newcommand{\f}[2]{\frac{#1}{#2}}
\newcommand{\p}[2]{\frac{\partial #1}{\partial #2}}
\providecommand{\t}[1]{\text{#1}}
\providecommand{\span}[1]{\text{span}\left(#1\right)}
\providecommand{\set}[1]{\left\{#1\right\}}
\providecommand{\l}[0]{\left}
\providecommand{\r}[0]{\right}
\newcommand{\m}[1]{\begin{matrix}#1\end{matrix}}
\newcommand{\bm}[1]{\begin{bmatrix}#1\end{bmatrix}}
\renewcommand{\bf}[1]{\mathbf{#1}}
\newcommand{\pn}[1]{\left( #1 \right)}
\newcommand{\abs}[1]{\left| #1 \right|}
\newcommand{\bk}[1]{\left[ #1 \right]}
\newcommand{\cis}[1]{\pn{\cos\pn{#1} + i\sin\pn{#1}}}
\newcommand{\cisi}[1]{\pn{\cos\pn{#1} - i\sin\pn{#1}}}
\renewcommand{\Im}[1]{\text{Im}\pn{#1}}
\renewcommand{\Re}[1]{\text{Re}\pn{#1}}
\renewcommand{\k}[0]{\f{1}{4\pi\epsilon_0}}
\renewcommand{\part}[1]{\vspace{1em}\noindent(#1)}

\makeatletter
\renewcommand*\env@matrix[1][*\c@MaxMatrixCols c]{%
  \hskip -\arraycolsep
  \let\@ifnextchar\new@ifnextchar
  \array{#1}}
\makeatother
\begin{document}
\problemlist{38-P5*, 35-E12, 35-P1, 35-E9}


\begin{problem}[38-P5*]
	A cube of edge $a$ has its edges parallel to the $x$, $y$, and $z$ axes of a rectangular coordinate system. A uniform electric field $\vec{\mathbf{E}}$ is parallel to the $y$ axis and a uniform magnetic field $\vec{\mathbf{B}}$ is parallel to the $x$ axis. Calculate
	\begin{enumerate}
		\item[(a)] the rate at which, according to the Poynting vector point of view, energy may be said to pass through each face of the cube and
		\item[(b)] the net rate at which the energy stored in the cube may be said to change. 
	\end{enumerate}	
\end{problem}
\begin{solution}
\end{solution}
\newpage


\begin{problem}[35-E12]
	The dipole moment associated with an atom of iron in an iron bar is $\SI{2.22}{\mu_B}$. Assume that all the atoms in the bar, which is \SI{4.86}{cm} long and has a cross-sectional area of \SI{1.31}{cm^2}, have their dipole moments aligned.
	\begin{enumerate}
		\item[(a)] What is the dipole moment of the bar?
		\item[(b)] What torque must be exerted to hold this magnet at right angles to an external field of \SI{1.53}{T}? 
	\end{enumerate}
\end{problem}
\begin{solution}
\end{solution}
\newpage

\begin{problem}[35-P1]
	A thin, plastic disk of radius $R$ has a charge $q$ uniformly distributed over its surface. If the disk rotates at an angular frequency $\omega$ about its axis, show that magnetic dipole moment of the disk is
	\[
		\mu = \f{\omega qR^2}{4}.
	\]
	(Hint: The rotating disk is equivalent to an array of current loops.)
\end{problem}
\begin{solution}
\end{solution}
\newpage

\begin{problem}[35-E9]
	In the lowest energy state of the hydrogen atom, the most probable distance between the single orbiting electron and the central proton is $\SI{5.29e-11}{m}$. Calculate
	\begin{enumerate}
		\item[(a)] the electric field and
		\item[(b)] the magnetic field 
	\end{enumerate}
	set up by the proton at this distance, measured along the proton's axis of spin. See Table 35-1 for the magnetic moment of the proton.
\end{problem}
\begin{solution}
\end{solution}

\end{document}