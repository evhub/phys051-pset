% HMC Math dept HW template example
% v0.04 by Eric J. Malm, 10 Mar 2005
\documentclass[12pt,letterpaper,boxed,cm]{hmcpset}

% set 1-inch margins in the document
\usepackage[margin=1in]{geometry}
\usepackage{mathtools}
\usepackage{mathrsfs}
% include this if you want to import graphics files with /includegraphics
\usepackage{graphicx}
\usepackage{cases}
\usepackage{hyperref}
\usepackage{siunitx}
\usepackage{tikz}
\usepackage{cases}
\usepackage{mathalfa}
\usetikzlibrary{arrows}

% info for header block in upper right hand corner
\name{Name: ~~~~~~~~~~~~~~~~~~~~~~~~~~~~~~~}
\class{Physics 51}
\assignment{Homework \#21}
\duedate{November 23, 2016}

\newcommand{\ev}[2]{\Big|_{#1}^{#2}}
\newcommand{\evv}[2]{\Big|_{#1}^{#2}}
\newcommand{\set}[1]{\left\{#1\right\}}
\newcommand{\s}[1]{\sqrt{#1}}
\newcommand{\f}[2]{\frac{#1}{#2}}
\newcommand{\p}[2]{\frac{\partial #1}{\partial #2}}
\providecommand{\t}[1]{\text{#1}}
\providecommand{\span}[1]{\text{span}\left(#1\right)}
\providecommand{\set}[1]{\left\{#1\right\}}
\providecommand{\l}[0]{\left}
\providecommand{\r}[0]{\right}
\newcommand{\m}[1]{\begin{matrix}#1\end{matrix}}
\newcommand{\bm}[1]{\begin{bmatrix}#1\end{bmatrix}}
\renewcommand{\bf}[1]{\mathbf{#1}}
\newcommand{\pn}[1]{\left( #1 \right)}
\newcommand{\abs}[1]{\left| #1 \right|}
\newcommand{\bk}[1]{\left[ #1 \right]}
\newcommand{\cis}[1]{\pn{\cos\pn{#1} + i\sin\pn{#1}}}
\newcommand{\cisi}[1]{\pn{\cos\pn{#1} - i\sin\pn{#1}}}
\renewcommand{\Im}[1]{\text{Im}\pn{#1}}
\renewcommand{\Re}[1]{\text{Re}\pn{#1}}
\renewcommand{\k}[0]{\f{1}{4\pi\epsilon_0}}
\renewcommand{\part}[1]{\vspace{1em}\noindent(#1)}

\makeatletter
\renewcommand*\env@matrix[1][*\c@MaxMatrixCols c]{%
  \hskip -\arraycolsep
  \let\@ifnextchar\new@ifnextchar
  \array{#1}}
\makeatother
\begin{document}
\problemlist{Townsend 1.\{4, 9, 13\}}

\begin{problem}[Townsend 1.4]
	A radio station broadcasts at a frequency $\nu =
\SI{91.5}{MHz}$ with a total radiated power of $P = \SI{20}{kW}$.
	\begin{enumerate}
		\item[(a)] What is the wavelength $\lambda$ of this radiation? 
		\item[(b)] What is the energy of each photon in \SI{}{eV}? How many photons are emitted each second? How many photons are emitted in each cycle? 
		\item[(c)] A particular radio receiver requires 2.0 microwatts of radiation to provide intelligible reception. How many \SI{91.5}{MHz} photons does this	 require per second? per cycle?
		\item[(d)] Do the answers to (b) and (c) indicate that the granularity of the electromagnetic radiation can be neglected in these circumstances?
	\end{enumerate}
\end{problem}
\begin{solution}
\end{solution}
\newpage

\begin{problem}[Townsend 1.9]
	A beam of UV light of wavelength $\lambda = \SI{197.0}{nm}$ falls onto a metal cathode. The stopping potential needed to keep any electrons from reaching the anode is \SI{2.08}{V}. 
	\begin{enumerate}
		\item[(a)] What is the work function W of the cathode surface, in eV? 
		\item[(b)] What is the velocity $v$ of the fastest electrons emitted from the cathode? \textit{Note}: Since $K_{\text{max}}/mc^2 \ll 1$, the nonrelativistic expression for the kinetic energy can be utilized here. 
		\item[(c)] If Avogadro's number of photons strikes each square meter of the surface in one hour, what is the average intensity $I$ of the beam, in units of \SI{}{W/m^2}?
	\end{enumerate}
\end{problem}
\begin{solution}
\end{solution}
\newpage

\begin{problem}[Townsend 1.13]
	 The maximum kinetic energy of electrons ejected from sodium is \SI{1.85}{eV} for radiation of \SI{300}{nm} and \SI{0.82}{eV} for radiation of \SI{400}{nm}. Use this data to determine Planck's constant and the work function of sodium.
\end{problem}
\begin{solution}
\end{solution}

\end{document}