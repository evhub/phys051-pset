% HMC Math dept HW template example
% v0.04 by Eric J. Malm, 10 Mar 2005
\documentclass[12pt,letterpaper,boxed,cm]{hmcpset}

% set 1-inch margins in the document
\usepackage[margin=1in]{geometry}
\usepackage{mathtools}
\usepackage{mathrsfs}
% include this if you want to import graphics files with /includegraphics
\usepackage{graphicx}
\usepackage{cases}
\usepackage{hyperref}
\usepackage{siunitx}
\usepackage{tikz}
\usetikzlibrary{arrows}

% info for header block in upper right hand corner
\name{Name: ~~~~~~~~~~~~~~~~~~~~~~~~~~~}
\class{Physics 51}
\assignment{Homework \#7}
\duedate{September 22, 2016}

\newcommand{\ev}[2]{\Big|_{#1}^{#2}}
\newcommand{\evv}[2]{\Big|_{#1}^{#2}}
\newcommand{\set}[1]{\left\{#1\right\}}
\newcommand{\s}[1]{\sqrt{#1}}
\newcommand{\f}[2]{\frac{#1}{#2}}
\newcommand{\p}[2]{\frac{\partial #1}{\partial #2}}
\providecommand{\t}[1]{\text{#1}}
\providecommand{\span}[1]{\text{span}\left(#1\right)}
\providecommand{\set}[1]{\left\{#1\right\}}
\providecommand{\l}[0]{\left}
\providecommand{\r}[0]{\right}
\newcommand{\m}[1]{\begin{matrix}#1\end{matrix}}
\newcommand{\bm}[1]{\begin{bmatrix}#1\end{bmatrix}}
\renewcommand{\bf}[1]{\mathbf{#1}}
\newcommand{\pn}[1]{\left( #1 \right)}
\newcommand{\abs}[1]{\left| #1 \right|}
\newcommand{\bk}[1]{\left[ #1 \right]}
\newcommand{\cis}[1]{\pn{\cos\pn{#1} + i\sin\pn{#1}}}
\newcommand{\cisi}[1]{\pn{\cos\pn{#1} - i\sin\pn{#1}}}
\renewcommand{\Im}[1]{\text{Im}\pn{#1}}
\renewcommand{\Re}[1]{\text{Re}\pn{#1}}
\renewcommand{\k}[0]{\f{1}{4\pi\epsilon_0}}

\makeatletter
\renewcommand*\env@matrix[1][*\c@MaxMatrixCols c]{%
  \hskip -\arraycolsep
  \let\@ifnextchar\new@ifnextchar
  \array{#1}}
\makeatother
\begin{document}
\problemlist{28-E42, SUP6}

\begin{problem}[28-E42]
	Consider two widely separated conducting spheres, 1 and 2, the second having twice the diameter of the first. The smaller sphere initially has a positive charge $q$ and the larger one is initially uncharged. You now connect the spheres with a long thin wire. 
\begin{enumerate}
	\item[(a)] How are the final potentials $V_1$ and $V_2$ of the spheres related?
	\item[(b)] Find the final charges $q_1$ and $q_2$ on the spheres in terms of $q$.
\end{enumerate}
\end{problem}
\begin{solution}
\end{solution}
\newpage


\begin{problem}[SUP6]
\begin{enumerate}
	\item[(a)] The current density across a cylindrical conductor of radius $R$ varies according to the equation
\[
	j = j_0\pn{1 - \f{r}{R}},
\]
where $r$ is the distance from the axis. Thus the current density is maximum $j_0$ at the axis $r = 0$ and decreases linearly to zero at the surface $r = R$. Calculate the current in terms of $j_0$ and the conductor's cross-sectional area $A = \pi R^2$. 
	\item[(b)] Suppose that, instead, the current density is a maximum $j_0$ at the surface and decreases linearly to zero at the axis, so that
\[
	j = j_0\pn{\f{r}{R}}.
\]
Calculate the current. Why is the result different from (a)?
\end{enumerate}
\end{problem}
\begin{solution}
\end{solution}

\end{document}
