% HMC Math dept HW template example
% v0.04 by Eric J. Malm, 10 Mar 2005
\documentclass[12pt,letterpaper,boxed,cm]{hmcpset}

% set 1-inch margins in the document
\usepackage[margin=1in]{geometry}
\usepackage{mathtools}
\usepackage{mathrsfs}
% include this if you want to import graphics files with /includegraphics
\usepackage{graphicx}
\usepackage{cases}
\usepackage{hyperref}
\usepackage{siunitx}
\usepackage{tikz}
\usepackage{cases}
\usepackage{mathalfa}
\usetikzlibrary{arrows}

% info for header block in upper right hand corner
\name{Name: ~~~~~~~~~~~~~~~~~~~~~~~~~~~~~~~}
\class{Physics 51}
\assignment{Homework \#16}
\duedate{November 3, 2016}

\newcommand{\ev}[2]{\Big|_{#1}^{#2}}
\newcommand{\evv}[2]{\Big|_{#1}^{#2}}
\newcommand{\set}[1]{\left\{#1\right\}}
\newcommand{\s}[1]{\sqrt{#1}}
\newcommand{\f}[2]{\frac{#1}{#2}}
\newcommand{\p}[2]{\frac{\partial #1}{\partial #2}}
\providecommand{\t}[1]{\text{#1}}
\providecommand{\span}[1]{\text{span}\left(#1\right)}
\providecommand{\set}[1]{\left\{#1\right\}}
\providecommand{\l}[0]{\left}
\providecommand{\r}[0]{\right}
\newcommand{\m}[1]{\begin{matrix}#1\end{matrix}}
\newcommand{\bm}[1]{\begin{bmatrix}#1\end{bmatrix}}
\renewcommand{\bf}[1]{\mathbf{#1}}
\newcommand{\pn}[1]{\left( #1 \right)}
\newcommand{\abs}[1]{\left| #1 \right|}
\newcommand{\bk}[1]{\left[ #1 \right]}
\newcommand{\cis}[1]{\pn{\cos\pn{#1} + i\sin\pn{#1}}}
\newcommand{\cisi}[1]{\pn{\cos\pn{#1} - i\sin\pn{#1}}}
\renewcommand{\Im}[1]{\text{Im}\pn{#1}}
\renewcommand{\Re}[1]{\text{Re}\pn{#1}}
\renewcommand{\k}[0]{\f{1}{4\pi\epsilon_0}}
\renewcommand{\part}[1]{\vspace{1em}\noindent(#1)}

\makeatletter
\renewcommand*\env@matrix[1][*\c@MaxMatrixCols c]{%
  \hskip -\arraycolsep
  \let\@ifnextchar\new@ifnextchar
  \array{#1}}
\makeatother
\begin{document}
\problemlist{38-E16, 38-E22, SUP3}

\begin{problem}[SUP3]
	\begin{enumerate}
		\item[(a)] Consider an electromagnetic wave in a vacuum with electric field $\vec{E} = E_0\hat{y}\sin\pn{kx-\omega t}$. What is the propagation direction of this electromagnetic wave?
		\item[(b)] Consider an electromagnetic wave with electric field $\vec{E} = E_0\hat{y}\sin\pn{kx+\omega t}$. What is the propagation direction of this electromagnetic wave?
		\item[(c)] Consider the electric field $\vec{E} = E_0\hat{y} \left[\sin\pn{kx-\omega t} + \sin\pn{kx+\omega t}\right]$. Show that the electric field satiesfies the wave equation
		\[
				\f{\partial^2\vec{E}}{\partial x^2} + 
				\f{\partial^2\vec{E}}{\partial y^2} + 
				\f{\partial^2\vec{E}}{\partial z^2} = 
				\f{1}{V^2}
				\f{\partial^2\vec{E}}{\partial t^2},
		\]
		provided the constants $k$ and $\omega$ are related as in part (a).
	\end{enumerate}
\end{problem}
\begin{solution}
\end{solution}
\newpage

\begin{problem}[38-E16]
The electric field associated with a plane electromagnetic wave is given by $E_x = 0$, $E_y = 0$, $E_z = E_0\sin k\pn{x-ct}$, where $E_0 = \SI{2.34e-4}{V/m}$ and $k = \SI{9.72e6}{m^{-1}}$. The wave is propagating in the $+x$ direction.
\begin{enumerate}
	\item[(a)] Write expressions for the components of the magnetic field of the wave.
	\item[(b)] Find the wavelength of the wave.
\end{enumerate}
\end{problem}
\begin{solution}
\end{solution}
\newpage

\begin{problem}[38-E22]
A plane electromagnetic wave is traveling in the negative $y$ direction. At a particular position and time, the magnetic field is along the positive $z$ axis and has a magnitude of $\SI{28}{\nano T}$. What are the direction and magnitude of the electric field at that position and at that time?
\end{problem}
\begin{solution}
\end{solution}
\end{document}